\section{Group Schemes}


\begin{definition}[Group-like elements]
  \label{0-grp-like}
  \uses{}
  \lean{Coalgebra.IsGroupLikeElem}
  \leanok

  An element $a$ of a coalgebra $A$ is \emph{group-like} if it is a unit and $\Delta(a) = a \ox a$, where $\Delta$ is the comultiplication map.
\end{definition}


\begin{lemma}[Bialgebra homs preserve group-like elements]
  \label{0-grp-like-map}
  \uses{0-grp-like}
  \lean{Coalgebra.IsGroupLikeElem.map}
  \leanok

  Let $f : A \to B$ be a bi-algebra hom. If $a \in A$ is group-like, then $f(a)$ is group-like too.
\end{lemma}
\begin{proof}
  \uses{}
  \leanok

  $a$ is a unit, so $f(a)$ is a unit too. Then
  \[
    f(a) \ox f(a) = (f \ox f)(\Delta_A(a)) = \Delta_B(f(a))
  \]
  so $f(a)$ is group-like.
\end{proof}


\begin{lemma}[Independence of group-like elements]
  \label{0-grp-like-lin-indep}
  \uses{0-grp-like}
  \lean{Coalgebra.linearIndepOn_isGroupLikeElem}
  \leanok

  The group-like elements in $A$ are linearly independent.
\end{lemma}
\begin{proof}
  \uses{0-tensor-lin-indep}
  \leanok

  See Lemma 4.23 in \cite{Milne_2017}.
\end{proof}


\begin{lemma}
  \label{0-grp-like-group-alg}
  \uses{0-grp-like}
  % \leanok

  The group-like elements of $k[M]$ are exactly the image of $M$.
\end{lemma}
\begin{proof}
  \uses{0-grp-like-lin-indep}
  % \leanok

  See Lemma 12.4 in \cite{Milne}.
\end{proof}


\begin{definition}
  \label{0-spec-grp-alg}
  \lean{AlgebraicGeometry.Spec, MonoidAlgebra}
  \leanok

  For a commutative group $M$ we define $D(M)$ as the spectrum $\Spec R[M]$ of the group algebra $R[M]$.
\end{definition}


\begin{proposition}
  \label{DM_func}
  \uses{0-spec-grp-alg}
  % \lean{}
  % \leanok

  For every finitely generated commutative group $M$, the algebraic group $D(M)$
  represents the functor $R \rightsquigarrow \Hom_{\mathsf{Grp}}(M,R^*)$.
\end{proposition}
\begin{proof}
  See Proposition 12.3 in \cite{Milne_2017}.
\end{proof}


\begin{proposition}
  \label{0-spec-grp-alg-basis}
  \uses{spec-grp-alg}
  % \lean{}
  % \leanok

  The choice of a basis for $M$ determines an isomorphism of $D(M)$
  with a finite product of copies of $\Gm$ and various $\mu_n$.
\end{proposition}
\begin{proof}
  See Proposition 12.3 in \cite{Milne_2017}.
\end{proof}


\begin{definition}
  \label{diag}
  \uses{0-grp-like}
  An algebraic group $G$ is \emph{diagonalizable}
  if the group-like elements in $\Gamma(G)$ span it as a $k$-vector space.
\end{definition}

\begin{theorem}
  \label{diag_iff_D}
  \uses{0-spec-grp-alg}
  \uses{diag}
  An algebraic group $G$ is diagonalizable
  if and only if it is isomorphic to $D(M)$ for some commutative group $M$.
\end{theorem}
\begin{proof}
  \uses{0-grp-like-lin-indep}
  See Theorem 12.8 in \cite{Milne_2017}.
\end{proof}

\begin{theorem}
  \label{congr_fggrp_diag}
  \uses{char_lattice}
  \uses{diag}
  The functor $M\rightsquigarrow D(M)$ is a contravariant equivalence
  from the category of finitely generated commutative groups to the category of
  diagonalizable algebraic groups (with quasi-inverse $G \rightsquigarrow X(G)$).
\end{theorem}
\begin{proof}
  \uses{diag_iff_D}
  \uses{0-spec-grp-alg-basis}
  See Theorem 12.9(a) in \cite{Milne_2017}. Case work required.
\end{proof}
