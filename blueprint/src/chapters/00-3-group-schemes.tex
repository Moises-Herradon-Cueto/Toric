\section{Group Schemes}


\subsection{Hopf algebras}


We want to show that Hopf algebras correspond to affine group schemes. This can easily be done
categorically assuming both categories on either side are defined thoughtfully. However, the
categorical version will not be workable with if we do not also have links to the non-categorical
notions. Therefore, we want to build the left, top and right edges of the following diagram so that
the bottom edge can be obtained by composing the three.

%    Cogrp Mod_R ->     Grp Sch_{Spec R}
%         |                    |
% R-Hopf algebra -> Group scheme over Spec R

First, let's do the left edge.


\begin{proposition}[Antipodes are antihomomorphisms]
  \label{0-hopf-antipode-antihom}
  \uses{}
  % \lean{}
  % \leanok

  The antipode map is anti-commutative.
\end{proposition}
\begin{proof}
  \uses{}
  % \leanok

  Any standard reference will have a proof.
\end{proof}


\begin{proposition}[Antipodes of commutative Hopf algebras are homomorphisms]
  \label{0-commhopf-antipode-hom}
  \uses{}
  % \lean{}
  % \leanok

  If $A$ is a commutative Hopf $R$-algebra, then the antipode is an $R$-algebra homomorphism.
\end{proposition}
\begin{proof}
  \uses{0-hopf-antipode-antihom}
  % \leanok

  Follows immediately from \ref{0-hopf-antipode-antihom} and commutativity.
\end{proof}


\begin{proposition}[Hopf algebras are cogroup objects in the category of algebras]
  \label{0-hopf-cogrp-alg}
  \uses{0-hopf-antipode-antihom}
  % \lean{}
  % \leanok

  From a $R$-Hopf algebra, one can build a cogroup object in the category of $R$-algebras.

  From a cogroup object in the category of $R$-algebras, one can build a $R$-Hopf algebra.
\end{proposition}
\begin{proof}
  \uses{}
  % \leanok

  Turn the arrows around.
\end{proof}


Second, let's do the top edge.


\begin{proposition}[Limit-preserving functors lift to over categories]
  \label{0-over-lim}
  \uses{}
  \lean{CategoryTheory.Limits.PreservesLimitsOfShape.overPost, CategoryTheory.Limits.PreservesLimitsOfSize.overPost, CategoryTheory.Limits.PreservesFiniteProducts.overPost}
  \leanok

  If $F : C \to D$ is a functor preserving limits of shape $J$, then so is the obvious functor $C / X \to D / F(X)$.
\end{proposition}
\begin{proof}
  \uses{}
  % \leanok

  Hopefully easy.
\end{proof}


\begin{proposition}[Fully faithful monoidal functors lift to group objects]
  \label{0-full-faithful-grp}
  \uses{}
  \lean{CategoryTheory.Functor.Faithful.mapGrp, CategoryTheory.Functor.Full.mapGrp}
  \leanok

  If a monoidal functor $F : C \to D$ is fully faithful, then so is $\Grp(F) : \Grp C \to \Grp D$.
\end{proposition}
\begin{proof}
  \uses{}
  \leanok

  Faithfulness is immediate.

  For fullness, assume $f : F(G) \to F(H)$ is a morphism. By fullness of $F$, find $g : G \to H$ such that $F(g) = f$. $g$ is a morphism because we can pull back each diagram from $D$ to $C$ along $F$ which is faithful.
\end{proof}


\begin{proposition}[Equivalences lift to group object categories]
  \label{0-grp-equiv}
  \uses{}
  \lean{CategoryTheory.Equivalence.mapGrp}
  \leanok

  If $e : C \backsimeq D$ is an equivalence of cartesian-monoidal categories, then $\Grp(e) : \Grp(C) \backsimeq \Grp(D)$ too is an equivalence of categories.
\end{proposition}
\begin{proof}
  \uses{}
  \leanok

  Transfer all diagrams.
\end{proof}


\begin{proposition}[The category of Hopf algebras is equivalent to the category of affine group schemes]
  \label{0-hopf-equiv-aff-grp-sch}
  \uses{}
  \lean{hopfAlgEquivAffGrpSch}
  \leanok

  The functor $A\rightsquigarrow \Spec A$ is a contravariant equivalence from the category of Hopf algebras over $R$ to the category of affine group schemes over $\Spec R$.
\end{proposition}
\begin{proof}
  \uses{0-over-lim, 0-grp-equiv}
  % \leanok

  $\Spec : \Ring \to \AffSch$ is a contravariant equivalence which preserves finite limits, hence so is
  $\Spec : \Ring_R \to \AffSch_{\Spec R}$ by Proposition \ref{0-over-lim}. Therefore we get an equivalence $\Grp(\Spec) : \Hopf_R \to \AffGrpSch_{\Spec R}$.
\end{proof}


The right edge is trivial. Finally, we obtain the bottom edge by composing the other three.


\begin{theorem}[Hopf algebras correspond to affine group schemes]
  \label{0-hopf-aff-grp-sch}
  \uses{}
  \lean{hopfSpec.instFull, hopfSpec.instFaithful, essImage_hopfSpec}
  \leanok

  If $A$ is a $R$-Hopf algebra, then $\Spec A$ is naturally an affine group scheme over $\Spec R$.

  If $G$ is an affine group scheme over $\Spec R$, then $\Gamma(G)$ is naturally an $R$-Hopf algebra.
\end{theorem}
\begin{proof}
  \uses{0-hopf-cogrp-alg, 0-hopf-equiv-aff-grp-sch}
  % \leanok

  Turn an unbundled Hopf algebra into a bundled one using Proposition \ref{0-hopf-cogrp-alg}. Turn
  the bundled Hopf algebra into an affine group scheme using Proposition \ref{0-hopf-equiv-aff-grp-sch}. By definition, an affine group scheme is a group scheme that is affine.

  For the other way around, follow the steps backwards.
\end{proof}


\subsection{Diagonalisable groups}


\begin{definition}[Group-like elements]
  \label{0-grp-like}
  \uses{}
  \lean{Coalgebra.IsGroupLikeElem}
  \leanok

  An element $a$ of a coalgebra $A$ is \emph{group-like} if it is a unit and $\Delta(a) = a \ox a$, where $\Delta$ is the comultiplication map.
\end{definition}


\begin{lemma}[Bialgebra homs preserve group-like elements]
  \label{0-grp-like-map}
  \uses{0-grp-like}
  \lean{Coalgebra.IsGroupLikeElem.map}
  \leanok

  Let $f : A \to B$ be a bi-algebra hom. If $a \in A$ is group-like, then $f(a)$ is group-like too.
\end{lemma}
\begin{proof}
  \uses{}
  \leanok

  $a$ is a unit, so $f(a)$ is a unit too. Then
  \[
    f(a) \ox f(a) = (f \ox f)(\Delta_A(a)) = \Delta_B(f(a))
  \]
  so $f(a)$ is group-like.
\end{proof}


\begin{lemma}[Independence of group-like elements]
  \label{0-grp-like-lin-indep}
  \uses{0-grp-like}
  \lean{Coalgebra.linearIndepOn_isGroupLikeElem}
  \leanok

  The group-like elements in $A$ are linearly independent.
\end{lemma}
\begin{proof}
  \uses{0-tensor-lin-indep}
  \leanok

  See Lemma 4.23 in \cite{Milne_2017}.
\end{proof}


\begin{lemma}
  \label{0-grp-like-grp-alg}
  \uses{0-grp-like}
  \lean{MonoidAlgebra.isGroupLikeElem_iff_mem_range_of}
  \leanok

  The group-like elements of $k[M]$ are exactly the image of $M$.
\end{lemma}
\begin{proof}
  \uses{0-grp-like-lin-indep}
  \leanok

  See Lemma 12.4 in \cite{Milne_2017}.
\end{proof}


\begin{definition}
  \label{0-spec-grp-alg}
  \lean{AlgebraicGeometry.Spec, MonoidAlgebra}
  \leanok

  For a commutative group $M$ we define $D_R(M)$ as the spectrum $\Spec R[M]$ of the group algebra $R[M]$.
\end{definition}


\begin{definition}
  \label{0-diag}
  \uses{0-spec-grp-alg}
  \lean{AlgebraicGeometry.Scheme.IsDiagonalisable}
  % \leanok

  An algebraic group $G$ over $\Spec R$ is {\bf diagonalisable}
  if it is isomorphic to $D_R(M)$ for some commutative group $M$.
\end{definition}


\begin{theorem}
  \label{0-diag-iff-grp-like-span}
  \uses{0-spec-grp-alg, 0-diag}
  \lean{AlgebraicGeometry.Scheme.isDiagonalisable_iff_span_isGroupLikeElem_eq_top}
  \leanok

  An algebraic group $G$ over a field $k$ is diagonalizable if and only if group-like elements span $\Gamma(G)$.
\end{theorem}
\begin{proof}
  \uses{0-grp-like-lin-indep}
  % \leanok

  See Theorem 12.8 in \cite{Milne_2017}.
\end{proof}


\begin{proposition}[The group algebra functor is fully faithful]
  \label{0-full-grp-hopf-alg}
  \uses{}
  % \lean{}
  % \leanok

  The functor $G \rightsquigarrow R[G]$ from the category of groups to the category of Hopf algebras over $R$ is fully faithful.
\end{proposition}
\begin{proof}
  \uses{0-diag-iff-grp-like-span, 0-grp-like-grp-alg, 0-grp-like-map}
  % \leanok

\end{proof}


\begin{theorem}
  \label{0-fg-comm-grp-equiv-diag-grp-sch}
  \uses{1-1-char-lat}
  \uses{0-diag}
  The functor $M\rightsquigarrow D(M)$ is a contravariant equivalence
  from the category of finitely generated commutative groups to the category of
  diagonalizable algebraic groups (with quasi-inverse $G \rightsquigarrow X(G)$).
\end{theorem}
\begin{proof}
  \uses{0-hopf-aff-grp-sch, 0-full-grp-hopf-alg}
  % \leanok

  The functor is essentially surjective by definition.
  It's faithfull since, $M\rightsquigarrow k[M]$ and $\Spec$ are.
  It's full by \ref{congr_hopf_affgrpsch} and \ref{0-full-grp-hopf-alg}.

  Also, see Theorem 12.9(a) in \cite{Milne_2017}.
\end{proof}
