\section{Group Schemes}


\begin{definition}[Group-like elements]
  \label{0-grp-like}
  \uses{}
  \lean{Coalgebra.IsGroupLikeElem}
  \leanok

  An element $a$ of a coalgebra $A$ is \emph{group-like} if it is a unit and $\Delta(a) = a \ox a$, where $\Delta$ is the comultiplication map.
\end{definition}


\begin{lemma}[Bialgebra homs preserve group-like elements]
  \label{0-grp-like-map}
  \uses{0-grp-like}
  \lean{Coalgebra.IsGroupLikeElem.map}
  \leanok

  Let $f : A \to B$ be a bi-algebra hom. If $a \in A$ is group-like, then $f(a)$ is group-like too.
\end{lemma}
\begin{proof}
  \uses{}
  \leanok

  $a$ is a unit, so $f(a)$ is a unit too. Then
  \[
    f(a) \ox f(a) = (f \ox f)(\Delta_A(a)) = \Delta_B(f(a))
  \]
  so $f(a)$ is group-like.
\end{proof}


\begin{lemma}[Independence of group-like elements]
  \label{0-grp-like-lin-indep}
  \uses{0-grp-like}
  \lean{Coalgebra.linearIndepOn_isGroupLikeElem}
  \leanok

  The group-like elements in $A$ are linearly independent.
\end{lemma}
\begin{proof}
  \uses{0-tensor-lin-indep}
  \leanok

  See Lemma 4.23 in \cite{Milne_2017}.
\end{proof}


\begin{lemma}
  \label{0-grp-like-grp-alg}
  \uses{0-grp-like}
  \lean{MonoidAlgebra.isGroupLikeElem_iff_mem_range_of}
  \leanok

  The group-like elements of $k[M]$ are exactly the image of $M$.
\end{lemma}
\begin{proof}
  \uses{0-grp-like-lin-indep}
  \leanok

  See Lemma 12.4 in \cite{Milne_2017}.
\end{proof}


\begin{definition}
  \label{0-spec-grp-alg}
  \lean{AlgebraicGeometry.Spec, MonoidAlgebra}
  \leanok

  For a commutative group $M$ we define $D_R(M)$ as the spectrum $\Spec R[M]$ of the group algebra $R[M]$.
\end{definition}


\begin{definition}
  \label{0-diag}
  \uses{0-spec-grp-alg}
  \lean{AlgebraicGeometry.Scheme.IsDiagonalisable}
  % \leanok

  An algebraic group $G$ over $\Spec R$ is {\bf diagonalisable}
  if it is isomorphic to $D_R(M)$ for some commutative group $M$.
\end{definition}


\begin{theorem}
  \label{0-diag-iff-grp-like-span}
  \uses{0-spec-grp-alg, 0-diag}
  \lean{AlgebraicGeometry.Scheme.isDiagonalisable_iff_span_isGroupLikeElem_eq_top}
  \leanok

  An algebraic group $G$ over a field $k$ is diagonalizable if and only if group-like elements span $\Gamma(G)$.
\end{theorem}
\begin{proof}
  \uses{0-grp-like-lin-indep}
  % \leanok

  See Theorem 12.8 in \cite{Milne_2017}.
\end{proof}


\begin{proposition}
  \label{0-full-grp-hopf}
  \uses{}
  % \lean{}
  % \leanok

  The functor $G \rightsquigarrow R[G]$ from the category of groups to the category of Hopf algebras over $R$ is fully faithful.
\end{proposition}
\begin{proof}
  \uses{0-diag-iff-grp-like-span, 0-grp-like-grp-alg, 0-grp-like-map}
  % \leanok

\end{proof}


\begin{proposition}[Finite-limit-preserving functors lift to over categories]
  \label{0-over-fin-lim}
  \uses{}
  \lean{CategoryTheory.Limits.PreservesFiniteProducts.overPost}
  \leanok

  If $F : C \to D$ is a functor preserving finite limits, then so is the obvious functor $C / x \to D / F(X)$.
\end{proposition}
\begin{proof}
  \uses{}
  % \leanok

  Hopefully easy.
\end{proof}


\begin{theorem}[Hopf algebras correspond to affine group schemes]
  \label{0-hopf-equiv-aff-grp-sch}
  \uses{0-over-fin-lim}
  \lean{specAsWeWantIt.instFull, specAsWeWantIt.instFaithful, essImage_hopfSpec}
  \leanok

  The functor $A\rightsquigarrow \Spec A$ is a contravariant equivalence
  from the category of Hopf algebras over $R$
  to the category of affine group schemes over $\Spec R$.
\end{theorem}

\begin{theorem}
  \label{congr_fggrp_diag}
  \uses{1-1-char-lat}
  \uses{0-diag}
  The functor $M\rightsquigarrow D(M)$ is a contravariant equivalence
  from the category of finitely generated commutative groups to the category of
  diagonalizable algebraic groups (with quasi-inverse $G \rightsquigarrow X(G)$).
\end{theorem}
\begin{proof}
  \uses{0-hopf-equiv-aff-grp-sch, 0-full-grp-hopf}
  % \leanok

  The functor is essentially surjective by definition.
  It's faithfull since, $M\rightsquigarrow k[M]$ and $\Spec$ are.
  It's full by \ref{congr_hopf_affgrpsch} and \ref{0-full-grp-hopf}.

  Also, see Theorem 12.9(a) in \cite{Milne_2017}.
\end{proof}
