\section{Cones and Affine Toric Varieties}

\subsection{Convex Polyhedral Cones}

Fix a pair of dual real vector spaces $M$ and $N$.


\begin{definition}[Convex cone generated by a set]
  \label{1-2-1-cone-hull}
  \uses{}
  \lean{Submodule.span}
  \leanok

  For a set $S \subseteq N$, the {\bf cone generated by $S$}, aka {\bf cone hull of $S$}, is
  $$\Cone(S) := \left\{\sum_{u \in S} \lambda_u | \lambda_u \ge 0\right\}$$
\end{definition}


\begin{definition}[Convex polyhedral cone]
  \label{1-2-1-polyhedral-cone}
  \uses{1-2-1-cone-hull}
  % \lean{}
  % \leanok

  A {\bf polyhedral cone} is a set that can be written as $\Cone(S)$ for some finite set $S$.
\end{definition}


\begin{definition}[Convex hull]
  \label{1-2-2-convex-hull}
  \uses{}
  \lean{convexHull}
  \leanok

  For a set $S \subseteq N$, the {\bf convex hull of $S$} is
  $$\Conv(S) := \left\{\sum_{u \in S} \lambda_u | \lambda_u \ge 0, \sum_u \lambda_u = 1\right\}$$
\end{definition}


\begin{definition}[Polytope]
  \label{1-2-2-polytope}
  \uses{1-2-2-convex-hull}
  \lean{IsPolytope}
  \leanok

  A {\bf polytope} is a set that can be written as $\Conv(S)$ for some finite set $S$.
\end{definition}


\subsection{Dual Cones and Faces}


\begin{definition}[Dual cone]
  \label{1-2-3-dual-cone}
  \uses{1-2-2-convex-hull}
  \lean{PointedCone.dual}
  \leanok

  Given a polyhedral cone $\sigma \subseteq N$, its {\bf dual cone} is defined by
  $$\sigma^\vee = \{m \in M | \forall u \in \sigma, \langle m, u\rangle \ge 0\}$$.
\end{definition}


\begin{proposition}[Dual of a polyhedral cone]
  \label{1-2-4-dual-polyhedral-cone}
  \uses{1-2-1-polyhedral-cone, 1-2-3-dual-cone}
  % \lean{}
  % \leanok

  If $\sigma$ is polyhedral, then its dual $\sigma^\vee$ is polyhedral too.
\end{proposition}
\begin{proof}
  \uses{}
  % \leanok

  Classic. See \cite{Oda_1988} maybe.
\end{proof}


\begin{proposition}[Dual of a sumset]
  \label{1-2-dual-cone-add}
  \uses{1-2-3-dual-cone}
  % \lean{}
  % \leanok

  If $\sigma_1, \sigma_2$ are two cones, then
  $$(\sigma_1 + \sigma_2)^\vee = \sigma_1^\vee \cap \sigma_2^\vee.$$
\end{proposition}
\begin{proof}
  \uses{}
  % \leanok

  Classic. See \cite{Oda_1988} maybe.
\end{proof}


\begin{proposition}[Double dual of a polyhedral cone]
  \label{1-2-4-double-dual-polyhedral-cone}
  \uses{1-2-1-polyhedral-cone, 1-2-3-dual-cone}
  % \lean{}
  % \leanok

  If $\sigma$ is polyhedral, then $\sigma^{\vee\vee} = \sigma$.
\end{proposition}
\begin{proof}
  \uses{}
  % \leanok

  Classic. See \cite{Oda_1988} maybe.
\end{proof}


Given $m \ne 0$ in $M$, we get the hyperplane
$$H_m = \{u \in N | \langle m, u\rangle = 0\} \subseteq N$$
and the closed half-space
$$H_m^+ = \{u \in N | \langle m, u\rangle \ge 0\} \subseteq N.$$


\begin{definition}[Face of a cone]
  \label{1-2-5-face}
  \uses{}
  \lean{IsExposed}
  \leanok

  If $\sigma$ is a polyhedral cone, then a subset of $\sigma$ is a {\bf face} iff it is the intersection of $\sigma$ with some halfspace. We write this $\tau \preceq \sigma$. If furthermore $\tau \ne \sigma$, we call $\tau$ a proper face and write $\tau \prec \sigma$.
\end{definition}


\begin{lemma}[Face of a polyhedral cone]
  \label{1-2-6-face-polyhedral-cone}
  \uses{1-2-1-polyhedral-cone, 1-2-5-face}
  % \lean{}
  % \leanok

  If $\sigma$ is a polyhedral cone, then every face of $\sigma$ is a polyhedral cone.
\end{lemma}


\begin{lemma}[Intersection of faces]
  \label{1-2-6-inter-faces}
  \uses{1-2-1-polyhedral-cone, 1-2-5-face}
  % \lean{}
  % \leanok

  If $\sigma$ is a polyhedral cone, then the intersection of two faces of $\sigma$ is a face of $\sigma$.
\end{lemma}
\begin{proof}
  \uses{}
  % \leanok

  Classic. See \cite{Oda_1988} maybe.
\end{proof}


\begin{lemma}[Face of a face]
  \label{1-2-6-face-face}
  \uses{1-2-1-polyhedral-cone, 1-2-5-face}
  % \lean{}
  % \leanok

  A face of a face of a polyhedral cone $\sigma$ is again a face of $\sigma$.
\end{lemma}
\begin{proof}
  \uses{}
  % \leanok

  Classic. See \cite{Oda_1988} maybe.
\end{proof}


\begin{lemma}
  \label{1-2-7-face-mem-of-add}
  \uses{1-2-1-polyhedral-cone, 1-2-5-face}
  % \lean{}
  % \leanok

  Let $\tau$ be a face of a polyhedral cone $\sigma$. If $v, w \in \sigma$ and $v + w \in \tau$, then $v, w \in \tau$.
\end{lemma}
\begin{proof}
  \uses{}
  % \leanok

  Classic. See \cite{Oda_1988} maybe.
\end{proof}


\begin{proposition}[Dual cone of the intersection of halfspaces]
  \label{1-2-8-dual-cone-inter-halfspaces}
  \uses{1-2-1-cone-hull, 1-2-3-dual-cone}
  % \lean{}
  % \leanok

  If $\sigma = H_{m_1}^+ \cap \dots \cap H_{m_s}^+$, then
  $$\sigma^\vee = \Cone(m_1, \dots, m_s).$$
\end{proposition}
\begin{proof}
  \uses{}
  % \leanok

  Classic. See \cite{Oda_1988} maybe.
\end{proof}


\begin{proposition}[Facets of a full dimensional cone]
  \label{1-2-8-facet-full-dim-cone}
  \uses{1-2-1-cone-hull, 1-2-5-face}
  % \lean{}
  % \leanok

  If $\sigma$ is a full dimensional cone, then facets of $\sigma$ are of the form $H_m \cap \sigma$.
\end{proposition}
\begin{proof}
  \uses{}
  % \leanok

  Classic. See \cite{Oda_1988} maybe.
\end{proof}


\begin{proposition}[Intersection of facets containing a face]
  \label{1-2-8-inter-facet}
  \uses{1-2-5-face}
  % \lean{}
  % \leanok

  Every proper face $\tau \prec \sigma$ of a polyhedral cone $\sigma$ is the intersection of the facets of $\sigma$ containing $\tau$.
\end{proposition}
\begin{proof}
  \uses{}
  % \leanok

  Classic. See \cite{Oda_1988} maybe.
\end{proof}


\begin{definition}[Dual face]
  \label{1-2-10-dual-face}
  \uses{1-2-3-dual-cone, 1-2-5-face}
  % \lean{}
  % \leanok

  Given a cone $\sigma$ and a face $\tau \preceq \sigma$, the {\bf dual face} to $\tau$ is
  $$\tau^* := \sigma^\vee \cap \tau^\perp$$
\end{definition}


\begin{proposition}[The dual face is a face of the dual]
  \label{1-2-10-dual-face-face-dual}
  \uses{1-2-10-dual-face}
  % \lean{}
  % \leanok

  If $\tau \preceq \sigma$, then $\tau^* \preceq \sigma^\vee$.
\end{proposition}
\begin{proof}
  \uses{}
  % \leanok

  Classic. See \cite{Oda_1988} maybe.
\end{proof}


\begin{proposition}[The double dual of a face]
  \label{1-2-10-double-dual-face-dual-face}
  \uses{1-2-10-dual-face}
  % \lean{}
  % \leanok

  If $\tau \preceq \sigma$, then $\tau^{**} = \tau$.
\end{proposition}
\begin{proof}
  \uses{1-2-4-double-dual-polyhedral-cone}
  % \leanok

  Classic. See \cite{Oda_1988} maybe.
\end{proof}


\begin{proposition}[The dual of a face is antitone]
  \label{1-2-10-dual-face-antitone}
  \uses{1-2-10-dual-face}
  % \lean{}
  % \leanok

  If $\tau' \preceq \tau \preceq \sigma$, then $\tau' \preceq \tau$.
\end{proposition}
\begin{proof}
  \uses{}
  % \leanok

  Classic. See \cite{Oda_1988} maybe.
\end{proof}


\begin{proposition}[The dimension of the dual of a face]
  \label{1-2-10-double-dual-face-dual-face}
  \uses{1-2-10-dual-face}
  % \lean{}
  % \leanok

  If $\tau \preceq \sigma$, then
  $$\dim \tau + \dim \tau^* = \dim N.$$
\end{proposition}
\begin{proof}
  \uses{}
  % \leanok

  Classic. See \cite{Oda_1988} maybe.
\end{proof}


\subsection{Relative Interiors}


\begin{definition}[Relative interior]
  \label{1-2-rel-interior}
  \uses{}
  \lean{intrinsicInterior}
  \leanok

  The {\bf relative interior}, aka {\bf intrinsic interior}, of a cone $\sigma$ is the interior of $\sigma$ as a subset of its span.
\end{definition}


\begin{lemma}[The relative interior in terms of the inner product]
  \label{1-2-rel-interior-inner}
  \uses{1-2-rel-interior}
  % \lean{}
  % \leanok

  For a cone $\sigma$,
  $$u \in \Relint(\sigma) \iff \forall m \in \sigma^\vee \setminus \sigma^\perp, \langle m, u\rangle > 0$$
\end{lemma}
\begin{proof}
  \uses{}
  % \leanok

  Classic. See \cite{Oda_1988} maybe.
\end{proof}


\begin{lemma}[Relative interior of a dual face]
  \label{1-2-rel-interior-dual-face}
  \uses{1-2-10-dual-face, 1-2-rel-interior}
  % \lean{}
  % \leanok

  If $\tau \preceq \sigma$ and $m \in \sigma^\vee$, then
  $$m \in \Relint(\tau^*) \iff \tau = H_m \cap \sigma$$
\end{lemma}
\begin{proof}
  \uses{}
  % \leanok

  Classic. See \cite{Oda_1988} maybe.
\end{proof}


\begin{lemma}[Minimal face of a cone]
  \label{1-2-min-face}
  \uses{1-2-5-face, 1-2-rel-interior}
  % \lean{}
  % \leanok

  If $\sigma$ is a cone, then $W := \sigma \cap (-\sigma)$ is a subspace. Furthermore,
  $W = H_m \cap \sigma$ whenever $m \in \Relint(\sigma^\vee)$.
\end{lemma}
\begin{proof}
  \uses{}
  % \leanok

  Classic. See \cite{Oda_1988} maybe.
\end{proof}


\subsection{Strong Convexity}


\begin{definition}[Salient cones]
  \label{1-2-12-salient-cone}
  \uses{}
  \lean{ConvexCone.Salient}
  \leanok

  A cone $\sigma$ is {\bf salient}, aka {\bf pointed} or {\bf strongly convex}, if $\sigma \cap (-\sigma) = \{0\}$.
\end{definition}


\begin{proposition}[Alternative definitions of salient cones]
  \label{1-2-12-salient-cone-tfae}
  \uses{1-2-3-dual-cone, 1-2-12-salient-cone}
  % \lean{}
  % \leanok

  The following are equivalent
  \begin{enumerate}
    \item $\sigma$ is salient
    \item $\{0\} \preceq \sigma$
    \item $\sigma$ contains no positive dimensional subspace
    \item $\dim \sigma^\vee = \dim N$
  \end{enumerate}
\end{proposition}
\begin{proof}
  \uses{}
  % \leanok

  Classic. See \cite{Oda_1988} maybe.
\end{proof}


\subsection{Separation}


\begin{lemma}[Separation lemma]
  \label{1-2-13-separation-lemma}
  \uses{1-2-1-polyhedral-cone, 1-2-5-face}
  % \lean{}
  % \leanok

  Let $\sigma_1, \sigma_2$ be polyhedral cones meeting along a common face $\tau$. Then
  $$\tau = H_m \cap \sigma_1 = H_m \cap \sigma_2$$
  for any $m \in \Relint(\sigma_1^\vee \cap (-\sigma_2)^\vee)$.
\end{lemma}
\begin{proof}
  \uses{1-2-dual-cone-add, 1-2-min-face}
  % \leanok

  See \cite{Cox_2011}.
\end{proof}


\subsection{Rational Polyhedral Cones}


Let $M$ and $N$ be dual lattices with associated vector spaces $M_\R := M \ox_\Z \R, N_\R := N \ox_\Z \R$.


\begin{definition}[Rational cone]
  \label{1-2-14-rat-cone}
  \uses{1-2-1-cone-hull}
  % \lean{}
  % \leanok

  A cone $\sigma \subseteq N_\R$ is {\bf rational} if $\sigma = \Cone(S)$ for some finite set $S \subseteq N$.
\end{definition}


\begin{lemma}[Faces of a rational cone]
  \label{1-2-14-face-rat-cone}
  \uses{}
  % \lean{}
  % \leanok

  If $\tau \preceq \sigma$ is a face of a rational cone, then $\tau$ itself is rational.
\end{lemma}
\begin{proof}
  \uses{}
  % \leanok

  Classic. See \cite{Oda_1988} maybe.
\end{proof}


\begin{lemma}[The dual of a rational cone]
  \label{1-2-14-dual-rat-cone}
  \uses{}
  % \lean{}
  % \leanok

  $\sigma^\vee$ is a rational cone iff $\sigma$ is.
\end{lemma}
\begin{proof}
  \uses{}
  % \leanok

  Classic. See \cite{Oda_1988} maybe.
\end{proof}


\begin{definition}[Ray generator]
  \label{1-2-ray-gen}
  \uses{1-2-14-rat-cone}
  % \lean{}
  % \leanok

  If $\rho$ is an edge of a rational cone $\sigma$, then the monoid $\rho \cap N$ is generated by a unique element $u_\rho \in \rho \cap N$, which we call the {\bf ray generator} of $\rho$.
\end{definition}


\begin{definition}[Minimal generators]
  \label{1-2-min-gen}
  \uses{1-2-ray-gen}
  % \lean{}
  % \leanok

  The {\bf minimal generators} of a rational cone $\sigma$ are the ray generators of its edges.
\end{definition}


\begin{lemma}[A rational cone is generated by its minimal generators]
  \label{1-2-15-cone-hull-min-gen}
  \uses{1-2-12-salient-cone, 1-2-min-gen}
  % \lean{}
  % \leanok

  A salient convex rational polyhedral cone is generated by its minimal generators.
\end{lemma}
\begin{proof}
  \uses{}
  % \leanok

  Classic. See \cite{Oda_1988} maybe.
\end{proof}


\begin{definition}[Regular cone]
  \label{1-2-16-reg-cone}
  \uses{1-2-min-gen}
  % \lean{}
  % \leanok

  A salient rational polyhedral cone $\sigma$ is {\bf regular}, aka {\bf smooth}, if its minimal generators form part of a $\Z$-basis of $N$.
\end{definition}


\begin{definition}[Simplicial cone]
  \label{1-2-16-simplicial-cone}
  \uses{1-2-min-gen}
  % \lean{}
  % \leanok

  A salient rational polyhedral cone $\sigma$ is {\bf simplicial} if its minimal generators are $\R$-linearly independent.
\end{definition}


\subsection{Semigroup Algebras and Affine Toric Varieties}


\begin{definition}[Dual lattice of a cone]
  \label{1-2-17-dual-lat-cone}
  \uses{1-2-3-dual-cone}
  % \lean{}
  % \leanok

  If $\sigma \subseteq N_\R$ is a polyhedral cone, then the lattice points
  \[
    S_\sigma := \sigma^\vee \cap M
  \]
  form a monoid.
\end{proposition}


\begin{proposition}[Gordan's lemma]
  \label{1-2-17-gordan-lemma}
  \uses{1-2-17-dual-lat-cone}
  % \lean{}
  % \leanok

  $S_\sigma$ is finitely generated as a monoid.
\end{proposition}
\begin{proof}
  \uses{1-2-14-dual-rat-cone}
  % \leanok

  See \cite{Cox_2011}.
\end{proof}


\begin{definition}[Affine toric variety of a rational polyhedral cone]
  \label{1-2-18-aff-tor-var-rat-polyhedral-cone}
  \uses{1-1-3-aff-tor-var, 1-2-17-dual-lat-cone}
  % \lean{}
  % \leanok

  $U_\sigma := \Spec \C[S_\sigma]$ is an affine toric variety.
\end{definition}


\begin{theorem}[Dimension of the affine toric variety of a rational polyhedral cone]
  \label{1-2-18-dim-aff-tor-var-rat-polyhedral-cone}
  \uses{1-2-12-salient-cone, 1-2-18-aff-tor-var-rat-polyhedral-cone}
  % \lean{}
  % \leanok

  \[
    \dim U_\sigma = \dim N \iff \text{ the torus of $U_\sigma$ is } T_N = N \ox_[\Z] \C^* \iff \sigma \text{ is salient}
  \]
\end{theorem}
\begin{proof}
  \uses{1-1-14-aff-tor-var-spec-group-alg, 1-2-12-salient-cone-tfae, 1-2-17-gordan-lemma}
  % \leanok

  See \cite{Cox_2011}.
\end{proof}


\begin{proposition}[The Hilbert basis of the dual lattice of a cone]
  \label{1-2-22-hilbert-basis}
  \uses{0-hilbert-basis, 1-2-min-gen, 1-2-12-salient-cone, 1-2-17-dual-lat-cone}
  % \lean{}
  % \leanok

  If $\sigma \subseteq N_\R$ is salient of maximal dimension, then the Hilbert basis $\mathcal H_{S_\sigma}$ contains the minimal generators of $\sigma^\vee$.
\end{proposition)
\begin{proof}
  \uses{0-hilbert-basis-finite}
  % \leanok

  See \cite{Cox_2011}.
\end{proof}
