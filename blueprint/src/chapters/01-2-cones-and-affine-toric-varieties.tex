\section{Cones and Affine Toric Varieties}

\subsection{Convex Polyhedral Cones}

Fix a pair of dual real vector spaces $M$ and $N$.


\begin{definition}[Convex cone generated by a set]
  \label{1-2-1-cone-hull}
  \uses{}
  \lean{Submodule.span}
  \leanok

  For a set $S \subseteq N$, the {\bf cone generated by $S$}, aka {\bf cone hull of $S$}, is
  $$\Cone(S) := \left\{\sum_{u \in S} \lambda_u | \lambda_u \ge 0\right\}$$
\end{definition}


\begin{definition}[Convex polyhedral cone]
  \label{1-2-1-polyhedral-cone}
  \uses{1-2-1-cone-hull}
  % \lean{}
  % \leanok

  A {\bf polyhedral cone} is a set that can be written as $\Cone(S)$ for some finite set $S$.
\end{definition}


\begin{definition}[Convex hull]
  \label{1-2-2-convex-hull}
  \uses{}
  \lean{convexHull}
  \leanok

  For a set $S \subseteq N$, the {\bf convex hull of $S$} is
  $$\Cone(S) := \left\{\sum_{u \in S} \lambda_u | \lambda_u \ge 0, \sum_u \lambda_u = 1\right\}$$
\end{definition}


\begin{definition}[Polytope]
  \label{1-2-2-polytope}
  \uses{1-2-2-convex-hull}
  % \lean{}
  % \leanok

  A {\bf polytope} is a set that can be written as $\Conv(S)$ for some finite set $S$.
\end{definition}


\subsection{Dual Cones and Faces}


\begin{definition}[Dual cone]
  \label{1-2-3-dual-cone}
  \uses{1-2-2-convex-hull}
  \lean{PointedCone.dual}
  \leanok

  Given a polyhedral cone $\sigma \subseteq N$, its {\bf dual cone} is defined by
  $$\sigma^\wedge = \{m \in M | \forall u \in \sigma, \langle m, u\rangle \ge 0\}$$.
\end{definition}


\begin{proposition}[Dual of a polyhedral cone]
  \label{1-2-4-dual-polyhedral-cone}
  \uses{1-2-1-polyhedral-cone, 1-2-2-dual-cone}
  % \lean{}
  % \leanok

  If $\sigma$ is polyhedral, then its dual $\sigma^\wedge$ is polyhedral too.
\end{proposition}
\begin{proof}
  \uses{}
  % \leanok

  Classic. See \cite{Oda_1988} maybe.
\end{proof}


\begin{proposition}[Double dual of a polyhedral cone]
  \label{1-2-4-double-dual-polyhedral-cone}
  \uses{1-2-1-polyhedral-cone, 1-2-2-dual-cone}
  % \lean{}
  % \leanok

  If $\sigma$ is polyhedral, then $\sigma^{\wedge\wedge} = \sigma$.
\end{proposition}
\begin{proof}
  \uses{}
  % \leanok

  Classic. See \cite{Oda_1988} maybe.
\end{proof}


Given $m \ne 0$ in $M$, we get the hyperplane
$$H_m = \{u \in N | \langle m, u\rangle = 0\} \subseteq N$$
and the closed half-space
$$H_m^+ = \{u \in N | \langle m, u\rangle \ge 0} \subseteq N.$$


\begin{definition}[Face of a cone]
  \label{1-2-5-face}
  \uses{}
  \lean{IsExposed}
  \leanok

  If $\sigma$ is a polyhedral cone, then a subset of $\sigma$ is a {\bf face} iff it is the intersection of $\sigma$ with some halfspace. We write this $\tau \preceq \sigma$. If furthermore $\tau \ne \sigma$, we call $\tau$ a proper face and write $\tau \prec \sigma$.
\end{definition}


\begin{lemma}[Face of a polyhedral cone]
  \label{1-2-6-face-polyhedral-cone}
  \uses{1-2-1-polyhedral-cone, 1-2-5-face}
  % \lean{}
  % \leanok

  If $\sigma$ is a polyhedral cone, then every face of $\sigma$ is a polyhedral cone.
\end{lemma}


\begin{lemma}[Intersection of faces]
  \label{1-2-6-inter-faces}
  \uses{1-2-1-polyhedral-cone, 1-2-5-face}
  % \lean{}
  % \leanok

  If $\sigma$ is a polyhedral cone, then the intersection of two faces of $\sigma$ is a face of $\sigma$.
\end{lemma}
\begin{proof}
  \uses{}
  % \leanok

  Classic. See \cite{Oda_1988} maybe.
\end{proof}


\begin{lemma}[Face of a face]
  \label{1-2-6-face-face}
  \uses{1-2-1-polyhedral-cone, 1-2-5-face}
  % \lean{}
  % \leanok

  A face of a face of a polyhedral cone $\sigma$ is again a face of $\sigma$.
\end{lemma}
\begin{proof}
  \uses{}
  % \leanok

  Classic. See \cite{Oda_1988} maybe.
\end{proof}


\begin{lemma}
  \label{1-2-7-face-mem-of-add}
  \uses{1-2-1-polyhedral-cone, 1-2-5-face}
  % \lean{}
  % \leanok

  Let $\tau$ be a face of a polyhedral cone $\sigma$. If $v, w \in \sigma$ and $v + w \in \tau$, then $v, w \in \tau$.
\end{lemma}
\begin{proof}
  \uses{}
  % \leanok

  Classic. See \cite{Oda_1988} maybe.
\end{proof}


\begin{proposition}[Dual cone of the intersection of halfspaces]
  \label{1-2-8-dual-cone-inter-halfspaces}
  \uses{1-2-1-cone-hull, 1-2-3-dual-cone}
  % \lean{}
  % \leanok

  If $\sigma = H_{m_1}^+ \cap \dots \cap H_{m_s}^+$, then
  $$\sigma^\wedge = \Cone(m_1, \dots, m_s).$$
\end{proposition}
\begin{proof}
  \uses{}
  % \leanok

  Classic. See \cite{Oda_1988} maybe.
\end{proof}


\begin{proposition}[Facets of a full dimensional cone]
  \label{1-2-8-facet-full-dim-cone}
  \uses{1-2-1-cone-hull, 1-2-5-face}
  % \lean{}
  % \leanok

  If $\sigma$ is a full dimensional cone, then facets of $\sigma$ are of the form $H_m \inter \sigma$.
\end{proposition}
\begin{proof}
  \uses{}
  % \leanok

  Classic. See \cite{Oda_1988} maybe.
\end{proof}


\begin{proposition}[Intersection of facets containing a face]
  \label{1-2-8-inter-facet}
  \uses{1-2-5-face}
  % \lean{}
  % \leanok

  Every proper face $\tau \prec \sigma$ of a polyhedral cone $\sigma$ is the intersection of the facets of $\sigma$ containing $\tau$.
\end{proposition}
\begin{proof}
  \uses{}
  % \leanok

  Classic. See \cite{Oda_1988} maybe.
\end{proof}


\begin{definition}[Dual face]
  \label{1-2-10-dual-face}
  \uses{1-2-3-dual-cone, 1-2-5-face}
  % \lean{}
  % \leanok

  Given a cone $\sigma$ and a face $\tau \preceq \sigma$, the {\bf dual face} to $\tau$ is
  $$\tau^* := \sigma^\wedge \cap \tau^\perp$$
\end{definition}


\begin{proposition}[The dual face is a face of the dual]
  \label{1-2-10-dual-face-face-dual}
  \uses{1-2-10-dual-face}
  % \lean{}
  % \leanok

  If $\tau \preceq \sigma$, then $\tau^* \preceq \sigma^\wedge$.
\end{proposition}
\begin{proof}
  \uses{}
  % \leanok

  Classic. See \cite{Oda_1988} maybe.
\end{proof}


\begin{proposition}[The double dual of a face]
  \label{1-2-10-double-dual-face-dual-face}
  \uses{1-2-10-dual-face}
  % \lean{}
  % \leanok

  If $\tau \preceq \sigma$, then $\tau^{**} = \tau$.
\end{proposition}
\begin{proof}
  \uses{1-2-4-double-dual-polyhedral-cone}
  % \leanok

  Classic. See \cite{Oda_1988} maybe.
\end{proof}


\begin{proposition}[The dual of a face is antitone]
  \label{1-2-10-dual-face-antitone}
  \uses{1-2-10-dual-face}
  % \lean{}
  % \leanok

  If $\tau' \preceq \tau \preceq \sigma$, then $\tau' \preceq \tau$.
\end{proposition}
\begin{proof}
  \uses{}
  % \leanok

  Classic. See \cite{Oda_1988} maybe.
\end{proof}


\begin{proposition}[The dimension of the dual of a face]
  \label{1-2-10-double-dual-face-dual-face}
  \uses{1-2-10-dual-face}
  % \lean{}
  % \leanok

  If $\tau \preceq \sigma$, then
  $$\dim \tau + \dim \tau^* = \dim N.$$
\end{proposition}
\begin{proof}
  \uses{}
  % \leanok

  Classic. See \cite{Oda_1988} maybe.
\end{proof}
