% In this file you should put the actual content of the blueprint.
% It will be used both by the web and the print version.
% It should *not* include the \begin{document}
%
% If you want to split the blueprint content into several files then
% the current file can be a simple sequence of \input. Otherwise It
% can start with a \section or \chapter for instance.

\setcounter{chapter}{-1} % Make Prerequisites be Chapter 0

\chapter{Prerequisites}

\section{Affine Monoids}


\begin{lemma}[Multivariate Laurent polynomials are an integral domain]
  \label{0-mv-laurent-poly-domain}
  \uses{}
  \lean{AddMonoidAlgebra.instIsDomain}
  \leanok

  Multivariate Laurent polynomials over an integral domain are an integral domain.
\end{lemma}
\begin{proof}
  \uses{}
  \leanok

  Come on.
\end{proof}


\begin{definition}[Affine monoid]
  \label{0-aff-mon}
  \lean{AddCancelCommMonoid, AddMonoid.FG, AddMonoid.IsTorsionFree}
  \leanok

  An \emph{affine monoid} is a finitely generated commutative monoid which is:
  \begin{itemize}
    \item cancellative: if $a + c = b + c$ then $a = b$, and
    \item torsion-free: if $n a = n b$ then $a = b$ (for $n \geq 1$).
  \end{itemize}
\end{definition}


\begin{proposition}[Embedding an affine monoid inside a lattice]
  \label{0-embed-aff-mon}
  \uses{0-aff-mon}
  \lean{AffineMonoid.embedding, AffineMonoid.embedding_injective}
  \leanok

  If $M$ is an affine monoid, then $M$ can be embedded inside $\Z^n$ for some $n$.
\end{proposition}
\begin{proof}
  \uses{}
  \leanok

  Embed $M$ inside its Grothendieck group $G$. Prove that $G$ is finitely generated free.
\end{proof}


\begin{proposition}[Affine monoid algebras are domains]
  \label{0-aff-mon-alg-domain}
  \uses{0-aff-mon}
  \lean{MonoidAlgebra.finiteType_of_fg}
  \leanok

  If $R$ is an integral domain $M$ is an affine monoid, then $R[M]$ is an integral domain and is a finitely generated $R$-algebra.
\end{proposition}
\begin{proof}
  \uses{0-mv-laurent-poly-domain, 0-embed-aff-mon}
  \leanok

  $i : R[M] \hookrightarrow R[\Z M]$ injects into an integral domain so is an integral domain. It's finitely generated by $\chi^{a_i}$ where $\mathcal A = \{a_1, \dotsc, a_s\}$ is a finite generating set for $M$.
\end{proof}


\begin{definition}[Irreducible element]
  \label{0-irred}
  \uses{}
  \lean{Irreducible}
  \leanok

  An element $x$ of a monoid $M$ is \emph{irreducible} if $x = y + z$ implies $y = 0$ or $z = 0$.
\end{definition}


\begin{proposition}[Irreducible elements lie in all sets generating a salient monoid]
  \label{0-irred-subset-gen}
  \uses{0-irred}
  \lean{addIrreducible_subset_of_addSubmonoidClosure_eq_top}
  \leanok

  If $M$ is a monoid with a single unit, and $S$ is a set generating $M$, then $S$ contains all irreducible elements of $M$.
\end{proposition}
\begin{proof}
  \uses{}
  \leanok

  Assume $p$ is an irreducible element. Since $S$ generates $M$, write
  \[
    p = \sum_i a_i
  \]
  where the $a_i$ are finitely many elements (not necessarily distinct) elements of $S$. Since $p$ is irreducible, we must have
  \[
    p = a_i \in S
  \]
  for some $i$.
\end{proof}


\begin{proposition}[A salient finitely generated monoid has finitely many irreducible elements]
  \label{0-irred-finite}
  \uses{0-irred}
  \lean{finite_addIrreducible}
  \leanok

  If $M$ is a finitely generated monoid with a single unit, then only finitely many elements of $M$ are irreducible.
\end{proposition}
\begin{proof}
  \uses{0-irred-subset-gen}
  \leanok

  Let $S$ be a finite set generating $M$. Write $I$ the set of irreducible elements. By Proposition \ref{0-irred-subset-gen}, $I \subseteq S$. Hence $I$ is finite.
\end{proof}


\begin{proposition}[A salient finitely generated cancellative monoid is generated by its irreducible elements]
  \label{0-irred-gen}
  \uses{0-irred}
  \lean{AddSubmonoid.closure_addIrreducible}
  \leanok

  If $M$ is a finitely generated cancellative monoid with a single unit, then $M$ is generated by its irreducible elements.
\end{proposition}
\begin{proof}
  \uses{}
  % \leanok

  We do not follow the proof from \cite{Cox_2011}.

  Let $S$ be a finite minimal generating set and assume for contradiction that $r \in S$ is reducible, say $r = a + b$ where $a, b$ are non-units. Write
  \[a = \sum_{s \in S} m_s s, b = \sum_{s \in S} n_s s\]
  for some $m_s, n_s \in \N$, so that
  \[r = \sum_{s \in S} (m_s + n_s) s.\]
  We distinguish three cases
  \begin{itemize}
    \item $m_r + n_r = 0$. Then
    \[r = \sum_{s \in S \setminus \{r\}} (m_s + n_s) s \in \langle S \setminus \{r\}\rangle\]
    contradicting the minimality of $S$.
    \item $m_r + n_r = 1$. Then
    \begin{align*}
      & 0 = \sum_{s \in S \setminus \{r\}} (m_s + n_s) s
      & \implies \forall s \in S \setminus \{r\}, m_s s = n_s s = 0
    \end{align*}
    Furthermore, either $m_r = 0$ or $n_r = 0$, so $a = 0$ or $b = 0$, contradicting the fact that $a$ and $b$ are non-units.
    \item $m_r + n_r \ge 2$. Then
    \[0 = r + \sum_{s \in S \setminus \{r\}} (m_s + n_s) s\]
    and $r = 0$, contradicting the minimality of $S$ once again.
  \end{itemize}
\end{proof}

\section{Tensor Product}


\begin{lemma}[The tensor product of linearly independent families]
  \label{0-tensor-lin-indep}
  \uses{}
  \lean{LinearIndependent.tmul}
  \leanok

  If $f$ and $g$ are linearly independent families of points in semimodules $M$ and $N$, then $i j \mapsto f i \ox g j$ is a linearly independent family of points in $M \ox N$.
\end{lemma}
\begin{proof}
  \uses{}
  % \leanok

  Assume
  \[
    \sum_{i, j} c_{i, j} f i \ox g j = \sum_{i, j} d_{i, j} f i \ox g j
  \]
  Then
  \[
    \sum_i f i \ox \left(\sum_j c_{i, j} g j\right) = \sum_i f i \ox \left(\sum_j d_{i, j} g j\right)
  \]
  Since $f$ is linearly independent,
  \[
    \sum_j c_{i, j} g j = \sum_j d_{i, j} g j
  \]
  for every $i$. Since $g$ is linearly independent, $c_{i, j} = d_{i, j}$ for all $i, j$, as wanted.
\end{proof}

\section{Group Schemes}


\begin{definition}[Group-like elements]
  \label{0-grp-like}
  \uses{}
  \lean{Coalgebra.IsGroupLikeElem}
  \leanok

  An element $a$ of a coalgebra $A$ is \emph{group-like} if it is a unit and $\Delta(a) = a \ox a$, where $\Delta$ is the comultiplication map.
\end{definition}


\begin{lemma}[Bialgebra homs preserve group-like elements]
  \label{0-grp-like-map}
  \uses{0-grp-like}
  \lean{Coalgebra.IsGroupLikeElem.map}
  \leanok

  Let $f : A \to B$ be a bi-algebra hom. If $a \in A$ is group-like, then $f(a)$ is group-like too.
\end{lemma}
\begin{proof}
  \uses{}
  \leanok

  $a$ is a unit, so $f(a)$ is a unit too. Then
  \[
    f(a) \ox f(a) = (f \ox f)(\Delta_A(a)) = \Delta_B(f(a))
  \]
  so $f(a)$ is group-like.
\end{proof}


\begin{lemma}[Independence of group-like elements]
  \label{0-grp-like-lin-indep}
  \uses{0-grp-like}
  \lean{Coalgebra.linearIndepOn_isGroupLikeElem}
  \leanok

  The group-like elements in $A$ are linearly independent.
\end{lemma}
\begin{proof}
  \uses{0-tensor-lin-indep}
  \leanok

  See Lemma 4.23 in \cite{Milne_2017}.
\end{proof}


\begin{lemma}
  \label{0-grp-like-grp-alg}
  \uses{0-grp-like}
  \lean{MonoidAlgebra.isGroupLikeElem_iff_mem_range_of}
  \leanok

  The group-like elements of $k[M]$ are exactly the image of $M$.
\end{lemma}
\begin{proof}
  \uses{0-grp-like-lin-indep}
  \leanok

  See Lemma 12.4 in \cite{Milne_2017}.
\end{proof}


\begin{definition}
  \label{0-spec-grp-alg}
  \lean{AlgebraicGeometry.Spec, MonoidAlgebra}
  \leanok

  For a commutative group $M$ we define $D_R(M)$ as the spectrum $\Spec R[M]$ of the group algebra $R[M]$.
\end{definition}


\begin{proposition}
  \label{DM_func}
  \uses{0-spec-grp-alg}
  % \lean{}
  % \leanok

  For every finitely generated commutative group $M$, the algebraic group $D(M)$
  represents the functor $R \rightsquigarrow \Hom_{\mathsf{Grp}}(M,R^*)$.
\end{proposition}
\begin{proof}
  See Proposition 12.3 in \cite{Milne_2017}.
\end{proof}


\begin{proposition}
  \label{0-spec-grp-alg-basis}
  \uses{0-spec-grp-alg}
  % \lean{}
  % \leanok

  The choice of a basis for $M$ determines an isomorphism of $D(M)$
  with a finite product of copies of $\Gm$ and various $\mu_n$.
\end{proposition}
\begin{proof}
  See Proposition 12.3 in \cite{Milne_2017}.
\end{proof}


\begin{definition}
  \label{diag}
  \uses{0-spec-grp-alg}
  \lean{AlgebraicGeometry.Scheme.IsDiagonalisable}
  \leanok

  An algebraic group $G$ over $\Spec R$ is {\bf diagonalisable}
  if it is isomorphic to $D_R(M)$ for some commutative group $M$.
\end{definition}

\begin{theorem}
  \label{0-diag-iff-grp-like-span}
  \uses{0-spec-grp-alg}
  \uses{diag}
  An algebraic group $G$ over a field $k$ is diagonalizable
  if and only if group-like elements span $\Gamma(G)$.
\end{theorem}
\begin{proof}
  \uses{0-grp-like-lin-indep}
  See Theorem 12.8 in \cite{Milne_2017}.
\end{proof}

\begin{theorem}
  \label{congr_fggrp_diag}
  \uses{char_lattice}
  \uses{diag}
  The functor $M\rightsquigarrow D(M)$ is a contravariant equivalence
  from the category of finitely generated commutative groups to the category of
  diagonalizable algebraic groups (with quasi-inverse $G \rightsquigarrow X(G)$).
\end{theorem}
\begin{proof}
  \uses{0-diag-iff-grp-like-span, 0-spec-grp-alg-basis, 0-grp-like-grp-alg}
  % \leanok

  See Theorem 12.9(a) in \cite{Milne_2017}. Case work required.
\end{proof}


\chapter{Affine Toric Varieties}

\section{Introduction to Affine Toric Varieties}

\begin{definition}
  \label{torusOver}
  \lean{torusOver}
  \leanok
  The (algebraic) torus $\Gm^n = \Spec(R[x_1^{\pm 1}, \dotsc, x_n^{\pm 1}])$ over $\Spec R$.
\end{definition}

\begin{definition}
  \label{char}
  Character: a group morphism $\chi : \Gm^n \to \Gm$.
\end{definition}

\begin{definition}
  \label{one_ps}
  One-parameter subgroup: a group morphism $\lambda : \Gm \to \Gm^n$.
\end{definition}

\begin{definition}
  \label{char_lattice}
  \uses{char}
  Character lattice $M = X(\Gm^n) := \Hom_{\mathsf{GrpSch}}(\Gm^n, \Gm)$.
\end{definition}

\begin{proposition}
  \label{charTor}
  \uses{char}
  $M = X(\Gm^n) \cong \Z^n$. For $m \in M$ we write $\chi^m$ for the corresponding character.
\end{proposition}

\begin{definition}
  \label{ops_lattice}
  \uses{one_ps}
  One-parameter subgroup/cocharacter lattice $N := \Hom(\Gm, \Gm^n)$.
\end{definition}

\begin{proposition}
  \label{prop:1.1.1.a}
  \uses{torusOver}
  Proposition 1.1.1(a): if $T_1, T_2$ are tori and $\Phi : T_1 \to T_2$ is a morphism which is a group homomorphism, then $\im \Phi$ is a closed subvariety which is a torus.
\end{proposition}

\begin{proposition}
  \label{prop:1.1.1.b}
  \uses{torusOver}
  Proposition 1.1.1(b): if $H \subseteq T$ is an irreducible subvariety which is a subgroup, then $H$ is a torus.
\end{proposition}

\begin{definition}
  \label{char_eigenspace}
  \uses{char}
  Character eigenspace
\end{definition}

\begin{proposition}
  \label{prop:1.1.2}
  \uses{char_eigenspace, torusOver}
  Proposition 1.1.2
\end{proposition}

\begin{definition}
  \label{char_ops_pairing}
  Character lattice and one-parameter subgroup pairing
\end{definition}

\begin{proposition}
  \uses{ops_lattice}
  $N = \Hom(M, \Z) \cong  \Z^n$. For $u \in N$ we write $\lambda^u$ for the corresponding cocharacter.
\end{proposition}
\begin{proof}
  \uses{charTor}
  \uses{char_ops_pairing}
\end{proof}

\begin{definition}
  \label{ToricVariety}
  \lean{ToricVariety}
  \leanok
  \uses{torusOver}
  A toric variety is a variety $X$ with
  \begin{itemize}
    \item an open embedding $T := (\C^\times)^n \hookrightarrow X$ with dense image
    \item such that the natural action $T \times T \to T$ of the torus on itself extends to an (algebraic) action $T \times X \to X$.
  \end{itemize}
\end{definition}

\begin{definition}
  \label{phiA}
  \uses{torusOver}
  \uses{char}
  Given a finite set $\mathcal A = \{a_1, \dotsc, a_s\} \subseteq M$, define $\Phi_{\mathcal{A}} : T \to \mathbb A^s$ given by $\Phi_{\mathcal A} (t) = (\chi^{a_1} (t), \dotsc, \chi^{a_s} (t))$.
\end{definition}

\begin{definition}
  \label{YA}
  \uses{phiA}
  $Y_{\mathcal{A}}$ is the (Zariski) closure of $\im \Phi_{\mathcal A}$ in $\mathbb A^s$.
\end{definition}

\begin{proposition}
  \label{prop:1.1.8}
  \uses{YA}
  \uses{char_lattice}
  Proposition 1.1.8
\end{proposition}
\begin{proof}
  \uses{prop:1.1.1.a}
\end{proof}

\begin{proposition}
  \label{prop:1.1.9}
  \uses{YA}
  Proposition 1.1.9
\end{proposition}

\begin{definition}
  \label{AddMonoidAlgebra.IsToricIdeal}
  \lean{AddMonoidAlgebra.IsToricIdeal}
  \leanok
  A lattice ideal $I_L = \langle x^\alpha - x^\beta |  \alpha, \beta \in \N^s \text{ and } \alpha - \beta \in L\rangle$, for some lattice $L \subseteq \Z^s$.

  A toric ideal is a prime lattice ideal.
\end{definition}

\begin{proposition}
  \label{AddMonoidAlgebra.isToricIdeal_iff_exists_span_single_sub_single}
  \uses{AddMonoidAlgebra.IsToricIdeal}
  \lean{AddMonoidAlgebra.isToricIdeal_iff_exists_span_single_sub_single}
  Proposition 1.1.11: an ideal is toric if and only if it's prime and generated by binomials $x^\alpha - x^\beta$.
\end{proposition}
\begin{proof}
  \uses{prop:1.1.1.b, prop:1.1.9}
\end{proof}

\begin{definition}
  \label{affSemi}
  \uses{AddCommMonoid}
  \uses{AddMonoid.FG}
  \uses{AddMonoid.IsTorsionFree}
  An affine monoid is a finitely generated submonoid of $(\Z^n, +)$. Equivalently it's a finitely generated commutative monoid which is:
  \begin{itemize}
    \item cancellative: if $a + c = b + c$ then $a = b$, and
    \item torsion-free: if $n a = n b$ then $a = b$ (for $n \geq 1$).
  \end{itemize}
\end{definition}

\begin{proposition}
  \label{prop:1.1.14}
  \uses{affSemi}
  \uses{ToricVariety}
  \uses{char_lattice}
  \uses{YA}
  If $S$ is an affine monoid, then
  \begin{enumerate}
    \item $\Bbbk[S]$ is an integral domain and is a finitely generated $\Bbbk$-algebra.
    \item $\Spec(\Bbbk[S])$ is an affine toric variety with character lattice $\Z S$.
    \item If $\mathcal A$ is a finite (monoid) generating set of $S$, then $\Spec(\Bbbk[S]) = Y_{\mathcal A}$.
  \end{enumerate}
\end{proposition}
\begin{proof}
  \uses{prop:1.1.8}
  $i : \Bbbk[S] \hookrightarrow \Bbbk[\Z S]$ injects into an integral domain so is an integral domain. It's finitely generated by $\chi^{a_i}$ where $\mathcal A = \{a_1, \dotsc, a_s\}$ is a finite generating set for $S$.

  $i$ induces a morphism $T \to \Spec(\Bbbk[S])$. It's an open embedding as $i$ gives the localization of $\Bbbk[S]$ at $\chi^{a_i}$, so $\im i$ is an affine open. It's dominant as $\Spec(\Bbbk[S])$ is integral and so is irreducible, and $\im i$ is open and nonempty, so dense. The torus action is given by the natural restriction of comultiplication on $\Bbbk[x_1^{\pm1}, \dotsc, x_n^{\pm1}]$.

  We get a $\Bbbk$-algebra homomorphism $\pi : \Bbbk[x_1, \dotsc, x_s] \to \Bbbk[\Z S]$ given by $\mathcal A$; this induces a morphism $\Phi_{\mathcal A} : T \to \Bbbk^s$. The kernel of $\pi$ is the toric ideal of $Y_{\mathcal A}$ and $\pi$ is clearly surjective, so $Y_{\mathcal A} = \mathbb V(\ker(\pi)) = \Spec(\Bbbk[x_1, \dotsc, x_s] / \ker(\pi)) = \Spec(\C[S])$.
\end{proof}

\begin{definition}
  \label{torActOnAlg}
  \uses{Torus}
  Torus action on semigroup algebra
\end{definition}

\begin{lemma}
  \label{lmm:1.1.16}
  \uses{torActOnAlg}
\end{lemma}
\begin{proof}
  \uses{prop:1.1.2}
\end{proof}

\begin{theorem}
  \label{thm:1.1.17}
  \uses{ToricVariety, YA, AddMonoidAlgebra.IsToricIdeal, affSemi}
  TFAE:
  \begin{enumerate}
    \item $V$ is an affine toric variety.
    \item $V = Y_{\mathcal A}$ for some finite $\mathcal A$.
    \item $V$ is an affine variety defined by a toric ideal.
    \item $V = \Spec \Bbbk[S]$ for an affine monoid $S$.
  \end{enumerate}
\end{theorem}
\begin{proof}
  \uses{torActOnAlg, prop:1.1.8, prop:1.1.9, prop:1.1.14, lmm:1.1.16}
\end{proof}

\section{Cones and Affine Toric Varieties}

\subsection{Convex Polyhedral Cones}

Fix a pair of dual real vector spaces $M$ and $N$.


\begin{definition}[Convex cone generated by a set]
  \label{1-2-1-cone-hull}
  \uses{}
  \lean{Submodule.span}
  \leanok

  For a set $S \subseteq N$, the {\bf cone generated by $S$}, aka {\bf cone hull of $S$}, is
  $$\Cone(S) := \left\{\sum_{u \in S} \lambda_u | \lambda_u \ge 0\right\}$$
\end{definition}


\begin{definition}[Convex polyhedral cone]
  \label{1-2-1-polyhedral-cone}
  \uses{1-2-1-cone-hull}
  % \lean{}
  % \leanok

  A {\bf polyhedral cone} is a set that can be written as $\Cone(S)$ for some finite set $S$.
\end{definition}


\begin{definition}[Convex hull]
  \label{1-2-2-convex-hull}
  \uses{}
  \lean{convexHull}
  \leanok

  For a set $S \subseteq N$, the {\bf convex hull of $S$} is
  $$\Conv(S) := \left\{\sum_{u \in S} \lambda_u | \lambda_u \ge 0, \sum_u \lambda_u = 1\right\}$$
\end{definition}


\begin{definition}[Polytope]
  \label{1-2-2-polytope}
  \uses{1-2-2-convex-hull}
  \lean{IsPolytope}
  \leanok

  A {\bf polytope} is a set that can be written as $\Conv(S)$ for some finite set $S$.
\end{definition}


\subsection{Dual Cones and Faces}


\begin{definition}[Dual cone]
  \label{1-2-3-dual-cone}
  \uses{1-2-2-convex-hull}
  \lean{PointedCone.dual}
  \leanok

  Given a polyhedral cone $\sigma \subseteq N$, its {\bf dual cone} is defined by
  $$\sigma^\vee = \{m \in M | \forall u \in \sigma, \langle m, u\rangle \ge 0\}$$.
\end{definition}


\begin{proposition}[Dual of a polyhedral cone]
  \label{1-2-4-dual-polyhedral-cone}
  \uses{1-2-1-polyhedral-cone, 1-2-3-dual-cone}
  % \lean{}
  % \leanok

  If $\sigma$ is polyhedral, then its dual $\sigma^\vee$ is polyhedral too.
\end{proposition}
\begin{proof}
  \uses{}
  % \leanok

  Classic. See \cite{Oda_1988} maybe.
\end{proof}


\begin{proposition}[Dual of a sumset]
  \label{1-2-dual-cone-add}
  \uses{1-2-3-dual-cone}
  % \lean{}
  % \leanok

  If $\sigma_1, \sigma_2$ are two cones, then
  $$(\sigma_1 + \sigma_2)^\vee = \sigma_1^\vee \cap \sigma_2^\vee.$$
\end{proposition}
\begin{proof}
  \uses{}
  % \leanok

  Classic. See \cite{Oda_1988} maybe.
\end{proof}


\begin{proposition}[Double dual of a polyhedral cone]
  \label{1-2-4-double-dual-polyhedral-cone}
  \uses{1-2-1-polyhedral-cone, 1-2-3-dual-cone}
  % \lean{}
  % \leanok

  If $\sigma$ is polyhedral, then $\sigma^{\vee\vee} = \sigma$.
\end{proposition}
\begin{proof}
  \uses{}
  % \leanok

  Classic. See \cite{Oda_1988} maybe.
\end{proof}


Given $m \ne 0$ in $M$, we get the hyperplane
$$H_m = \{u \in N | \langle m, u\rangle = 0\} \subseteq N$$
and the closed half-space
$$H_m^+ = \{u \in N | \langle m, u\rangle \ge 0\} \subseteq N.$$


\begin{definition}[Face of a cone]
  \label{1-2-5-face}
  \uses{}
  \lean{IsExposed}
  \leanok

  If $\sigma$ is a cone, then a subset of $\sigma$ is a {\bf face} iff it is the intersection of $\sigma$ with some halfspace. We write this $\tau \preceq \sigma$. If furthermore $\tau \ne \sigma$, we call $\tau$ a proper face and write $\tau \prec \sigma$.
\end{definition}


\begin{definition}[Edge of a cone]
  \label{1-2-5-edge}
  \uses{1-2-5-face}
  % \lean{}
  % \leanok

  A dimension 1 face of a cone is called an \emph{edge}.
\end{definition}


\begin{definition}[Facet of a cone]
  \label{1-2-5-facet}
  \uses{1-2-5-face}
  % \lean{}
  % \leanok

  A codimension 1 face of a cone is called a \emph{facet}.
\end{definition}


\begin{lemma}[Face of a polyhedral cone]
  \label{1-2-6-face-polyhedral-cone}
  \uses{1-2-1-polyhedral-cone, 1-2-5-face}
  % \lean{}
  % \leanok

  If $\sigma$ is a polyhedral cone, then every face of $\sigma$ is a polyhedral cone.
\end{lemma}


\begin{lemma}[Intersection of faces]
  \label{1-2-6-inter-faces}
  \uses{1-2-1-polyhedral-cone, 1-2-5-face}
  % \lean{}
  % \leanok

  If $\sigma$ is a polyhedral cone, then the intersection of two faces of $\sigma$ is a face of $\sigma$.
\end{lemma}
\begin{proof}
  \uses{}
  % \leanok

  Classic. See \cite{Oda_1988} maybe.
\end{proof}


\begin{lemma}[Face of a face]
  \label{1-2-6-face-face}
  \uses{1-2-1-polyhedral-cone, 1-2-5-face}
  % \lean{}
  % \leanok

  A face of a face of a polyhedral cone $\sigma$ is again a face of $\sigma$.
\end{lemma}
\begin{proof}
  \uses{}
  % \leanok

  Classic. See \cite{Oda_1988} maybe.
\end{proof}


\begin{lemma}
  \label{1-2-7-face-mem-of-add}
  \uses{1-2-1-polyhedral-cone, 1-2-5-face}
  % \lean{}
  % \leanok

  Let $\tau$ be a face of a polyhedral cone $\sigma$. If $v, w \in \sigma$ and $v + w \in \tau$, then $v, w \in \tau$.
\end{lemma}
\begin{proof}
  \uses{}
  % \leanok

  Classic. See \cite{Oda_1988} maybe.
\end{proof}


\begin{proposition}[Dual cone of the intersection of halfspaces]
  \label{1-2-8-dual-cone-inter-halfspaces}
  \uses{1-2-1-cone-hull, 1-2-3-dual-cone}
  % \lean{}
  % \leanok

  If $\sigma = H_{m_1}^+ \cap \dots \cap H_{m_s}^+$, then
  $$\sigma^\vee = \Cone(m_1, \dots, m_s).$$
\end{proposition}
\begin{proof}
  \uses{}
  % \leanok

  Classic. See \cite{Oda_1988} maybe.
\end{proof}


\begin{proposition}[Facets of a full dimensional cone]
  \label{1-2-8-facet-full-dim-cone}
  \uses{1-2-1-cone-hull, 1-2-5-facet}
  % \lean{}
  % \leanok

  If $\sigma$ is a full dimensional cone, then facets of $\sigma$ are of the form $H_m \cap \sigma$.
\end{proposition}
\begin{proof}
  \uses{}
  % \leanok

  Classic. See \cite{Oda_1988} maybe.
\end{proof}


\begin{proposition}[Intersection of facets containing a face]
  \label{1-2-8-inter-facet}
  \uses{1-2-5-facet}
  % \lean{}
  % \leanok

  Every proper face $\tau \prec \sigma$ of a polyhedral cone $\sigma$ is the intersection of the facets of $\sigma$ containing $\tau$.
\end{proposition}
\begin{proof}
  \uses{}
  % \leanok

  Classic. See \cite{Oda_1988} maybe.
\end{proof}


\begin{definition}[Dual face]
  \label{1-2-10-dual-face}
  \uses{1-2-3-dual-cone, 1-2-5-face}
  % \lean{}
  % \leanok

  Given a cone $\sigma$ and a face $\tau \preceq \sigma$, the {\bf dual face} to $\tau$ is
  $$\tau^* := \sigma^\vee \cap \tau^\perp$$
\end{definition}


\begin{proposition}[The dual face is a face of the dual]
  \label{1-2-10-dual-face-face-dual}
  \uses{1-2-10-dual-face}
  % \lean{}
  % \leanok

  If $\tau \preceq \sigma$, then $\tau^* \preceq \sigma^\vee$.
\end{proposition}
\begin{proof}
  \uses{}
  % \leanok

  Classic. See \cite{Oda_1988} maybe.
\end{proof}


\begin{proposition}[The double dual of a face]
  \label{1-2-10-double-dual-face-dual-face}
  \uses{1-2-10-dual-face}
  % \lean{}
  % \leanok

  If $\tau \preceq \sigma$, then $\tau^{**} = \tau$.
\end{proposition}
\begin{proof}
  \uses{1-2-4-double-dual-polyhedral-cone}
  % \leanok

  Classic. See \cite{Oda_1988} maybe.
\end{proof}


\begin{proposition}[The dual of a face is antitone]
  \label{1-2-10-dual-face-antitone}
  \uses{1-2-10-dual-face}
  % \lean{}
  % \leanok

  If $\tau' \preceq \tau \preceq \sigma$, then $\tau' \preceq \tau$.
\end{proposition}
\begin{proof}
  \uses{}
  % \leanok

  Classic. See \cite{Oda_1988} maybe.
\end{proof}


\begin{proposition}[The dimension of the dual of a face]
  \label{1-2-10-double-dual-face-dual-face}
  \uses{1-2-10-dual-face}
  % \lean{}
  % \leanok

  If $\tau \preceq \sigma$, then
  $$\dim \tau + \dim \tau^* = \dim N.$$
\end{proposition}
\begin{proof}
  \uses{}
  % \leanok

  Classic. See \cite{Oda_1988} maybe.
\end{proof}


\subsection{Relative Interiors}


\begin{definition}[Relative interior]
  \label{1-2-rel-interior}
  \uses{}
  \lean{intrinsicInterior}
  \leanok

  The {\bf relative interior}, aka {\bf intrinsic interior}, of a cone $\sigma$ is the interior of $\sigma$ as a subset of its span.
\end{definition}


\begin{lemma}[The relative interior in terms of the inner product]
  \label{1-2-rel-interior-inner}
  \uses{1-2-rel-interior}
  % \lean{}
  % \leanok

  For a cone $\sigma$,
  $$u \in \Relint(\sigma) \iff \forall m \in \sigma^\vee \setminus \sigma^\perp, \langle m, u\rangle > 0$$
\end{lemma}
\begin{proof}
  \uses{}
  % \leanok

  Classic. See \cite{Oda_1988} maybe.
\end{proof}


\begin{lemma}[Relative interior of a dual face]
  \label{1-2-rel-interior-dual-face}
  \uses{1-2-10-dual-face, 1-2-rel-interior}
  % \lean{}
  % \leanok

  If $\tau \preceq \sigma$ and $m \in \sigma^\vee$, then
  $$m \in \Relint(\tau^*) \iff \tau = H_m \cap \sigma$$
\end{lemma}
\begin{proof}
  \uses{}
  % \leanok

  Classic. See \cite{Oda_1988} maybe.
\end{proof}


\begin{lemma}[Minimal face of a cone]
  \label{1-2-min-face}
  \uses{1-2-5-face, 1-2-rel-interior}
  % \lean{}
  % \leanok

  If $\sigma$ is a cone, then $W := \sigma \cap (-\sigma)$ is a subspace. Furthermore,
  $W = H_m \cap \sigma$ whenever $m \in \Relint(\sigma^\vee)$.
\end{lemma}
\begin{proof}
  \uses{}
  % \leanok

  Classic. See \cite{Oda_1988} maybe.
\end{proof}


\subsection{Strong Convexity}


\begin{definition}[Salient cones]
  \label{1-2-12-salient-cone}
  \uses{}
  \lean{ConvexCone.Salient}
  \leanok

  A cone $\sigma$ is {\bf salient}, aka {\bf pointed} or {\bf strongly convex}, if $\sigma \cap (-\sigma) = \{0\}$.
\end{definition}


\begin{proposition}[Alternative definitions of salient cones]
  \label{1-2-12-salient-cone-tfae}
  \uses{1-2-3-dual-cone, 1-2-12-salient-cone}
  % \lean{}
  % \leanok

  The following are equivalent
  \begin{enumerate}
    \item $\sigma$ is salient
    \item $\{0\} \preceq \sigma$
    \item $\sigma$ contains no positive dimensional subspace
    \item $\dim \sigma^\vee = \dim N$
  \end{enumerate}
\end{proposition}
\begin{proof}
  \uses{}
  % \leanok

  Classic. See \cite{Oda_1988} maybe.
\end{proof}


\subsection{Separation}


\begin{lemma}[Separation lemma]
  \label{1-2-13-separation-lemma}
  \uses{1-2-1-polyhedral-cone, 1-2-5-face}
  % \lean{}
  % \leanok

  Let $\sigma_1, \sigma_2$ be polyhedral cones meeting along a common face $\tau$. Then
  $$\tau = H_m \cap \sigma_1 = H_m \cap \sigma_2$$
  for any $m \in \Relint(\sigma_1^\vee \cap (-\sigma_2)^\vee)$.
\end{lemma}
\begin{proof}
  \uses{1-2-dual-cone-add, 1-2-min-face}
  % \leanok

  See \cite{Cox_2011}.
\end{proof}


\subsection{Rational Polyhedral Cones}


Let $M$ and $N$ be dual lattices with associated vector spaces $M_\R := M \ox_\Z \R, N_\R := N \ox_\Z \R$.


\begin{definition}[Rational cone]
  \label{1-2-14-rat-cone}
  \uses{1-2-1-cone-hull}
  % \lean{}
  % \leanok

  A cone $\sigma \subseteq N_\R$ is {\bf rational} if $\sigma = \Cone(S)$ for some finite set $S \subseteq N$.
\end{definition}


\begin{lemma}[Faces of a rational cone]
  \label{1-2-14-face-rat-cone}
  \uses{1-2-5-face, 1-2-14-rat-cone}
  % \lean{}
  % \leanok

  If $\tau \preceq \sigma$ is a face of a rational cone, then $\tau$ itself is rational.
\end{lemma}
\begin{proof}
  \uses{}
  % \leanok

  Classic. See \cite{Oda_1988} maybe.
\end{proof}


\begin{lemma}[The dual of a rational cone]
  \label{1-2-14-dual-rat-cone}
  \uses{1-2-3-dual-cone, 1-2-14-rat-cone}
  % \lean{}
  % \leanok

  $\sigma^\vee$ is a rational cone iff $\sigma$ is.
\end{lemma}
\begin{proof}
  \uses{}
  % \leanok

  Classic. See \cite{Oda_1988} maybe.
\end{proof}


\begin{definition}[Ray generator]
  \label{1-2-ray-gen}
  \uses{1-2-5-edge, 1-2-14-rat-cone}
  % \lean{}
  % \leanok

  If $\rho$ is an edge of a rational cone $\sigma$, then the monoid $\rho \cap N$ is generated by a unique element $u_\rho \in \rho \cap N$, which we call the {\bf ray generator} of $\rho$.
\end{definition}


\begin{definition}[Minimal generators]
  \label{1-2-min-gen}
  \uses{1-2-ray-gen}
  % \lean{}
  % \leanok

  The {\bf minimal generators} of a rational cone $\sigma$ are the ray generators of its edges.
\end{definition}


\begin{lemma}[A rational cone is generated by its minimal generators]
  \label{1-2-15-cone-hull-min-gen}
  \uses{1-2-12-salient-cone, 1-2-min-gen}
  % \lean{}
  % \leanok

  A salient convex rational polyhedral cone is generated by its minimal generators.
\end{lemma}
\begin{proof}
  \uses{}
  % \leanok

  Classic. See \cite{Oda_1988} maybe.
\end{proof}


\begin{definition}[Regular cone]
  \label{1-2-16-reg-cone}
  \uses{1-2-min-gen}
  % \lean{}
  % \leanok

  A salient rational polyhedral cone $\sigma$ is {\bf regular}, aka {\bf smooth}, if its minimal generators form part of a $\Z$-basis of $N$.
\end{definition}


\begin{definition}[Simplicial cone]
  \label{1-2-16-simplicial-cone}
  \uses{1-2-min-gen}
  % \lean{}
  % \leanok

  A salient rational polyhedral cone $\sigma$ is {\bf simplicial} if its minimal generators are $\R$-linearly independent.
\end{definition}


\subsection{Semigroup Algebras and Affine Toric Varieties}


\begin{definition}[Dual lattice of a cone]
  \label{1-2-17-dual-lat-cone}
  \uses{1-2-3-dual-cone}
  % \lean{}
  % \leanok

  If $\sigma \subseteq N_\R$ is a polyhedral cone, then the lattice points
  \[
    S_\sigma := \sigma^\vee \cap M
  \]
  form a monoid.
\end{definition}


\begin{proposition}[Gordan's lemma]
  \label{1-2-17-gordan-lemma}
  \uses{1-2-17-dual-lat-cone}
  % \lean{}
  % \leanok

  $S_\sigma$ is finitely generated as a monoid.
\end{proposition}
\begin{proof}
  \uses{1-2-14-dual-rat-cone}
  % \leanok

  See \cite{Cox_2011}.
\end{proof}


\begin{definition}[Affine toric variety of a rational polyhedral cone]
  \label{1-2-18-aff-tor-var-rat-polyhedral-cone}
  \uses{1-1-3-aff-tor-var, 1-2-17-dual-lat-cone}
  % \lean{}
  % \leanok

  $U_\sigma := \Spec \C[S_\sigma]$ is an affine toric variety.
\end{definition}


\begin{theorem}[Dimension of the affine toric variety of a rational polyhedral cone]
  \label{1-2-18-dim-aff-tor-var-rat-polyhedral-cone}
  \uses{1-2-12-salient-cone, 1-2-18-aff-tor-var-rat-polyhedral-cone}
  % \lean{}
  % \leanok

  \[
    \dim U_\sigma = \dim N \iff \text{ the torus of $U_\sigma$ is } T_N = N \ox_[\Z] \C^* \iff \sigma \text{ is salient}
  \]
\end{theorem}
\begin{proof}
  \uses{1-1-14-aff-tor-var-spec-group-alg, 1-2-12-salient-cone-tfae, 1-2-17-gordan-lemma}
  % \leanok

  See \cite{Cox_2011}.
\end{proof}


\begin{proposition}[The Hilbert basis of the dual lattice of a cone]
  \label{1-2-22-hilbert-basis}
  \uses{0-hilbert-basis, 1-2-min-gen, 1-2-12-salient-cone, 1-2-17-dual-lat-cone}
  % \lean{}
  % \leanok

  If $\sigma \subseteq N_\R$ is salient of maximal dimension, then the Hilbert basis $\mathcal H_{S_\sigma}$ contains the minimal generators of $\sigma^\vee$.
\end{proposition}
\begin{proof}
  \uses{0-hilbert-basis-finite}
  % \leanok

  See \cite{Cox_2011}.
\end{proof}

% \section{Properties of Affine Toric Varieties}


% \chapter{Projective Toric Varieties}

% \subsection{Lattice Points and Projective Toric Varieties}

% \subsection{Lattice Points and Polytopes}

% \section{Polytopes and Projective Toric Varieties}

% \section{Properties of Projective Toric Varieties}


% \chapter{Normal Toric Varieties}

% \section{Fans and Normal Toric Varieties}

% \subsection{The Orbit-Cone Correspondence}

% \subsection{Toric Morphisms}

% \subsection{Complete and Proper}


\bibliographystyle{plain} % We choose the "plain" reference style
\bibliography{Toric}

